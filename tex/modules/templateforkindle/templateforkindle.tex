\documentclass[oneside,10pt,showtrims]{memoir}
\setstocksize{110.985mm}{84.560mm}
\settrimmedsize{110.985mm}{84.560mm}{*}
\settrims{0pt}{0pt}
\setlrmarginsandblock{3mm}{*}{*}
\setulmarginsandblock{3mm}{*}{*}
\setlength{\parskip}{0pt}
\setheadfoot{0mm}{0mm}
\setheaderspaces{*}{0mm}{*}
\setmarginnotes{0mm}{0mm}{0mm}
\checkandfixthelayout
\frenchspacing
\sloppy
\pagestyle{empty}

\title{Para una tumba sin nombre}
\author{JUAN CARLOS ONETTI}
\date{1959}

\usepackage[spanish]{babel}
%Begin font
\usepackage{fontspec}
\setmainfont{DanteMTStd-Medium}
%\linespread{1.05} % Use if need more leading (space between lines)
\usepackage{scalefnt} %Use to change font size
%End font

%Begin Decorative packages
\usepackage{lettrine}
\setcounter{DefaultLines}{3}
\usepackage{pifont}
\usepackage{fourier-orns}
\usepackage{color}

%End decorative packages

%Begin title Style
\makeatletter
\def\thickhrulefill{\leavevmode \leaders \hrule height 1pt\hfill \kern \z@}
\def\maketitle{%
\null
\thispagestyle{empty}%
\vskip 1cm
\begin{flushright}
\normalfont\Large \@author
\end{flushright}
\vfil
\hrule height 2pt
\par
\begin{center}
\Huge\strut \@title \par
\end{center}
\hrule height 2pt
\par
\vfil
\vfil
\null
\cleardoublepage
}
\makeatother
%\author{Isidore Ducasse, Comte de Lautréamont}
%\author{Lautréamont}
%\title{Les Chants de Maldoror}
%\date{1874}
%End Title Style

%Begin Chapter Style
\makeatletter
\newcommand\thickhrulefillm{\leavevmode \leaders \hrule height 1ex \hfill \kern \z@}
\setlength\midchapskip{10pt}
\makechapterstyle{VZ14}{
\renewcommand\chapternamenum{}
\renewcommand\printchaptername{}
\renewcommand\chapnamefont{\Large\scshape}
\renewcommand\printchapternum{%
\chapnamefont\null\thickhrulefillm\quad
\@chapapp\space\thechapter\quad\thickhrulefillm}
\renewcommand\printchapternonum{%
\par\thickhrulefillm\par\vskip\midchapskip
\hrule\vskip\midchapskip
}
\renewcommand\chaptitlefont{\Huge\scshape\centering}
\renewcommand\afterchapternum{%
\par\nobreak\vskip\midchapskip\hrule\vskip\midchapskip}
\renewcommand\afterchaptertitle{%
\par\vskip\midchapskip\hrule\nobreak\vskip\afterchapskip}
}
\makeatother
\chapterstyle{VZ14}
%End Chapter Style

%Begin Invisible trim
\definecolor{light-gray}{gray}{0.95}
\renewcommand*{\trimmarkscolor}{\color{light-gray}}
%End Invisible Point

\begin{document}

\scalefont{1.020}
\maketitle
\mainmatter
\chapter{I}
\aliaspagestyle{chapter}{empty}
\pagestyle{empty}
\lettrine[slope=0pt,findent=0em,nindent=-.4em]{T}{}
\small ODOS NOSOTROS\normalsize \scalefont{1.020}, los notables, los que tenemos derecho a jugar al póker en el Club Progreso y a dibujar iniciales con entumecida vanidad al pie de las cuentas por copas o comidas en el Plaza. Todos nosotros sabemos cómo es un entierro en Santa María. Algunos fuimos, en su oportunidad, el mejor amigo de la familia; se nos ofreció el privilegio de ver la cosa desde un principio y, además, el privilegio de iniciarla.

Es mejor, más armonioso, que la cosa empiece de noche, después y antes del sol. Fuimos a lo de Miramonte o a lo de Grimm, «Cochería Suiza». A veces, hablo de los veteranos, podíamos optar; otras, la elección se había decidido en rincones de la casa de duelo, por una razón, por diez o por ninguna. Yo, cuando puedo, elijo a Grimm para las familias viejas. Se sienten más cómodas con la brutalidad o indiferencia de Grimm, que insiste en hacer personalmente todo lo indispensable y lo que inventa por capricho. Prefieren al viejo por motivos raciales, esto puede verlo cualquiera; pero yo he visto además que agradecen su falta de hipocresía, el alivio que les proporciona enfrentando a la muerte como un negocio, considerando al cadáver como un simple bulto transportable.
\renewcommand{\pfbreakdisplay}{%
\decofourright\quad\decofourright\quad\decofourright}
\fancybreak{\pfbreakdisplay}
Hemos ido, casi siempre en la madrugada, serios pero cómodos en la desgracia, con una premeditada voz varonil y no cautelosa, a golpear en la puerta eternamente iluminada de Miramonte o de Grimm. Miramonte, en cambio, confía todo, en apariencia, a los empleados y se dedica, vestido de negro, peinado de negro, con su triste bigote negro y el brillo discretamente equívoco de los ojos de mulato, a mezclarse entre los dolientes, a estrechar manos y difundir consuelos. Esto les gusta a los otros, a los que no tuvieron abuelos arando en la colonia; también los he visto. Golpeamos, golpeo bajo el letrero luminoso violeta y explico mi misión a uno de los dos, al gringo o al mulato; cualquiera de ellos la conocía cinco minutos después del último suspiro y aguardaba. Grimm bosteza, se pone los anteojos y abre un libro enorme.
\end{document}
