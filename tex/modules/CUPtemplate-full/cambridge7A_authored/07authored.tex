% 07authored.tex
% 2011/02/28, v3.00 gamma

\chapter{Exercises}
\label{rarities}

\section{Organizing}
Exercises can be handled
in more than one way, as an enunciation or within a list, depending on your style and preference.
They can be scattered through the book, or organised in sets at the end of sections or
chapters, or some combination. But if you mix up styles we strongly recommend
you give the different types different names, for example, Exercises could be
scattered in the text, and Problems could be organised into sets, or vice versa.

\subsection{Scattered through the text --  exer or exer*}
There are two ways of handling exercises scattered through a chapter.
\begin{enumerate}
\item Use amsthm to define an \verb"exer" or \verb"exer*" environment
subject to \verb"\theoremstyle{definition}". See Chapter~\ref{mathchap} for details.
These environments must be defined in the root file for this document.
Exercises created with \verb"exer" are numbered, if at all, in the same sequence as theorems etc.
\item Use the \verb"exerciselist" environment, described below, with a single item.
Exercises created within this environment will be numbered in a
sequence separate from that for theorems etc.

\end{enumerate}

\subsection{At the end of sections -- exerciselist}
The \cambridge\ class file defines the \verb"exerciselist" environment
for setting lists of numbered exercises at the end of sections. These
will not automatically be gathered under a heading, so there will be no  mention of them
in the Table of Contents. Therefore you may wish to list them under a \verb"\subsection"
and set the heading depth appropriately or use the
appropriate \verb"\addcontentsline" command.

There is an option for adding a label such as `Exercise' or `Problem'. The code
\begin{verbatim}
  \begin{exerciselist}[Exercise]
     \item Show that the link between shock formation and
       film rupture is invoked here because of the \ldots
     \item Show that the physical interpretation of \ldots
       \label{physi-ex}
  \end{exerciselist}
\end{verbatim}
will produce
  \begin{exerciselist}[Exercise]
     \item Show that the link between shock formation and
       film rupture is invoked here because of the \ldots
     \item Show that the physical interpretation of \ldots
       \label{physi-ex}
  \end{exerciselist}
Like other numbered environments, individual exercises
(e.g. Exercise~\ref{physi-ex}) can be labeled for automatic cross-referencing.

\subsection{At the end of chapters -- exercises}
If you are gathering all exercises at the end of a given chapter,
use the \verb"exercises" environment rather than \verb"exerciselist". This environment generates an entry
in the table of contents and starts a new unnumbered section and running head. For example,
\begin{verbatim}
  \begin{exercises}
    \item Let the film thickness be $h_0$,
          \begin{equation}
            h=h_0 H{\xi}.
          \label{exerciseeq}
          \end{equation}
          Integrating once, \ldots
    \item Assuming the flow far away from \ldots
  \end{exercises}
\end{verbatim}
will produce (note the mention in the Table of Contents!)
  \begin{exercises}
    \item Let the film thickness be $h_0$,
          \begin{equation}
            h=h_0 H{\xi}.
          \label{exerciseeq}
          \end{equation}
          Integrating once, \ldots
    \item Assuming the flow far away from \ldots
  \end{exercises}
%
If appropriate, you may change the `Exercises' heading to one of the following:
%
\begin{enumerate}[(iii)]
\item `Exercise' -- by using \verb"\begin{exercise}...\end{exercise}"
\item `Problems' -- by using \verb"\begin{problems}...\end{problems}"
\item `Problem' -- by using \verb"\begin{problem}...\end{problem}"
\end{enumerate}
%
For instance,
\begin{verbatim}
  \begin{problems}
    \item By treating $y$ as the independent variable,
          show that the general solution of \ldots
    \item An electrical circuit contains a resistance \ldots
          \label{circuit}
  \end{problems}
\end{verbatim}
will typeset the following:
  \begin{problems}
    \item By treating $y$ as the independent variable,
          show that the general solution of \ldots
    \item An electrical circuit contains a resistance \ldots
          \label{circuit}
  \end{problems}


\endinput
