% author_preface.tex
% 2011/02/28, v3.00 gamma

\chapter*{Preface}
This guide is for authors preparing a book for
Cambridge University Press using \LaTeX\ and the \cambridge\ class
file. It assumes you have some familiarity with \LaTeX\ --
preferably with book.cls, which is itself somewhat different from article.cls.
It is not a substitute for the \LaTeX\ manual itself.


The \cambridge\  class file preserves the standard \LaTeX\ interface,
so any document that can be produced using the standard
{\LaTeX}2e book.cls can also be produced with {\cambridge}.cls.
However, the measure (i.e. width of text) for {\cambridge}.cls
is different from that for book.cls, so
line breaks will change and tables, figures and
long equations may need adjusting if you've already
used book.cls to create a draft.
Commands that differ from the standard \LaTeX\ interface,
or that are provided in addition to the standard interface,
are documented below.

This guide was created by processing the following
(the full root file is in Appendix \ref{root}:
\begin{verbatim}
  \documentclass[spanningrule]{cambridge7A}% options
     \usepackage{natbib}
     \usepackage[figuresright]{rotating}
     \usepackage{floatpag}
     \rotfloatpagestyle{empty}
     \usepackage{amsthm}
     \usepackage{graphicx}
     \usepackage{txfonts}
     \usepackage[scaled=0.9]{couriers}
     \usepackage{multind}\ProvidesPackage{multind}
      :
\end{verbatim}

Even if your book does not use references, rotated items,
computer code, theorems, graphics, or multiple
indexes, it will not hurt to include the packages above.
If you include \verb"multind.sty",
you must also insert \verb"\ProvidesPackage{multind}"; this command sends a message
to the class file to restyle the index into the \cambridge\ style.

Don't use the following standard document class options:
\begin{itemize}
\item \verb"10pt, 11pt, 12pt";
\item \verb"oneside" (\verb"twoside" is the default);
\item \verb"fleqn, leqno, titlepage, twocolumn".
\end{itemize}


\section*{A word about style}
If you so wish, the source files for this guide can be used as templates
for (parts of) your book. It's a really good idea
to observe good programming style -- after all,
\LaTeX\ is a programming language. Make sure for example, that
you list all of your definitions and commands in the preamble,
and that you don't include any that never get used. Don't duplicate them.
Don't use different macros to do the same job. Don't overwrite them
without cause; if you need locally to
\verb"\renewcommand", then make sure you revert back to the original command
as soon as you can.  Make sure the lines in
your root file are short: note there's a difference between line feed and carriage return
in some text editors. Don't include lots of local page make-up commands
unless you're producing final files for printing, or unless you
need to do so for float control. Structure your document
using the environments or commands provided rather than sticking in \verb"\vspace" followed
by some text in bold, for example. Don't number displayed items to which you're not
going to refer. If you are going to refer to things, then use \verb"\label",
\verb"\ref", \verb"\cite", etc. When
you make a decision, document it in the root file, so you can refer back to
it during the writing of your
book  (which can take place over several years!).
For the same reason, keep a style sheet in
which you list things like your hyphenation or capitalization rules.
Most importantly, \emph{be consistent in the way you typeset your book}.

All the above will make the writing, editing, copyediting, correction and reformatting of your
book much more manageable.

\section*{Workflow}
At some stage in the writing of your book, certainly before it's finished, you should discuss
with your CUP editor how the production of your book will be handled. We need to know:
 is the book being prepared
by you in its final design; who is imposing final design or inputting copyeditorial corrections;
when and how will the index be compiled; will the book be printed from final files provided by you;
how competent in \LaTeX\ are you? The answers to these questions will help determine the workflow
your book will follow during production. In any event, before the book is finished,
you should supply your editor with a sample file for evaluating and testing.


Note: books, and chapters,  must carry copyright lines if they are to be
posted on personal or institutional webpages.


