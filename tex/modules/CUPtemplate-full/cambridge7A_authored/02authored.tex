% 02authored.tex
% 2011/02/28, v3.00 gamma

  \chapter{Numbering and headings}
  \label{chapstructure}


\section{Chapter numbering}
Chapter numbers are generated automatically when the full book
is compiled with all chapters in place. Unnumbered chapters, such as the preface,
are specified using the \verb"\chapter*" command.

\section{Section numbering}

\LaTeX\ provides five levels of section heads, and all are defined
in the \cambridge\ class file: \verb"\section", \verb"\subsection",
 \verb"\subsubsection", \verb"\paragraph", and \verb"\subparagraph".
The first three levels are numbered, unless you use a starred version
such as \verb"\section*".

If your book includes an unnumbered chapter (e.g. \verb"\chapter*{Introduction}",
then ensure that all the numbered elements within that chapter
(e.g. section heads, equations, figures, etc) are unnumbered,
by using \verb"\section*{...}" for example.
Otherwise, sections will be numbered 0.1, 0.2, etc.
The same applies to headings subsidiary to an unnumbered section heading,
e.g. subsections, or items that are numbered by section.

\section{Running heads}
In \cambridge\ books, running heads are
\begin{itemize}
\item chapter titles on even-numbered pages (versos), and
\item section numbers (if they exist) and titles  on odd-numbered pages (rectos).
\end{itemize}

If the chapter or section title is long, a shorter version for the running
head can be specified using an optional argument
to the \verb"\chapter" or \verb"\section" command, for example:
\begin{verbatim}
  \chapter[Running head title]{Full chapter title}
\end{verbatim}

To ensure that the full versions of chapter and section titles are
given in the table of contents, simply do the following:

\begin{verbatim}
  \chapter[TOC entry]{Full chapter title}
  \chaptermark{Short title, i.e., running head entry}

  \section[TOC entry]{Full section title
    \sectionmark{Short title, i.e. running head entry}}
    \sectionmark{Short title, i.e. running head entry}
\end{verbatim}
The TOC entry may in fact be the same as the full chapter or section title.
But note that for sections, you need the optional argument to \verb"\section",
even if `TOC entry' is in fact the same text as `Full section title'.
Also, the \verb"\sectionmark" has to be entered twice as shown, because
the first \verb"\sectionmark" deals with the header of the page that
the \verb"\section" command falls on, and the second deals with subsequent pages.
However, there's no need to include section number in the \verb"\sectionmark" argument.

\section{Parts}
Sometimes you may wish break the book into segments that are bigger than chapters. For this
you can use the \verb"\part" command. This will create a Part Title page which will always
appear on an odd-numbered page the verso of which will be blank. An entry of the Table of Contents
will be created automatically: parts are numbered in `words'. For example

\begin{verbatim}
\part{The First Part}
\end{verbatim}

\noindent
produces the Part Title on page \pageref{partpage}. Use max caps for Part Titles, as here.
It's good style to enter Part Titles in the root file. 

\section{Other}
Numbering of other items, such as equations, figures and tables, theorems etc., references, exercises, are
dealt with in relevant chapters.

\endinput